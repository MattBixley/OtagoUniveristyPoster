%%%%%%%%%%%%%%%%%%%%%%%%%%%%%%%%%%%%%%%%%%%%%%%%%%%%%%%%%%%%%%%%%%%%
%% %%	Posterdown PDF class for LaTeX files	 08-JAN-2019
%% %%	For any information please send an e-mail to:
%% %%		brentthonre18@gmail.com (Brent Thorne)
%% %%
%% %%	Initial class provided by:
%% %%		Brent Thorne
%% %% Contributors: Shea Connell (SC)
%%%%%%%%%%%%%%%%%%%%%%%%%%%%%%%%%%%%%%%%%%%%%%%%%%%%%%%%%%%%%%%%%%%%

\documentclass[article,6pt,extrafontsizes]{memoir}

%utf-8 seems to be important
\RequirePackage[utf8]{inputenc}
\RequirePackage[T1]{fontenc}
\RequirePackage{lmodern}
\RequirePackage{multicol}
\RequirePackage{graphicx}
\RequirePackage{lipsum}
\RequirePackage{blindtext}
\RequirePackage[svgnames,table]{xcolor}
\RequirePackage{tikz}
\RequirePackage[framemethod=tikz]{mdframed}
\RequirePackage{color}
\RequirePackage{geometry}
\RequirePackage{adjmulticol}
\RequirePackage[skins,most,listings,skins]{tcolorbox}

%For kable extra package :)
\RequirePackage{booktabs}
\RequirePackage{longtable}
\RequirePackage{array}
\RequirePackage{multirow}
\RequirePackage{wrapfig}
\RequirePackage{float}
\RequirePackage{colortbl}
\RequirePackage{pdflscape}
\RequirePackage{pagecolor}
\RequirePackage{tabu}
\RequirePackage{threeparttable}
\RequirePackage{threeparttablex}
\RequirePackage[normalem]{ulem}
\RequirePackage{makecell}
\RequirePackage{wrapfig}

%rof hyperrefs
\RequirePackage{hyperref}
\hypersetup{
    colorlinks=true,
    linkcolor=linkcol,
    citecolor=citecol,
    filecolor=linkcol,
    urlcolor=urlcol,
}
%For figure and table placement
\RequirePackage{float}
\floatplacement{figure}{H}
\floatplacement{table}{H}

%%%%%%%%% COLOURS %%%%%%%%
%Fill/ Line Colours
\definecolor{titleboxbgcol}{HTML}{00508F}
\definecolor{titleboxbordercol}{HTML}{F9C000}
\definecolor{columnlinecol}{HTML}{008080}
\definecolor{bodybgcol}{HTML}{ffffff}
\definecolor{sectitlebgcol}{HTML}{00508F}
\definecolor{sectitlebordercol}{HTML}{F9C000}
% Text Colours
\definecolor{titletextcol}{HTML}{ffffff}
\definecolor{authortextcol}{HTML}{1b1c20}
\definecolor{affiliationtextcol}{HTML}{FFFFFF}
\definecolor{sectitletextcol}{HTML}{ffffff}
\definecolor{bodytextcol}{HTML}{000000}
\definecolor{footnotetextcol}{HTML}{ffffff}
\definecolor{citecol}{HTML}{CC0000}
\definecolor{urlcol}{HTML}{008080}
\definecolor{linkcol}{HTML}{008080}


%Memoir spacing options
%spacing between figure/ table and caption
\setlength{\abovecaptionskip}{0.4in}
\setlength{\belowcaptionskip}{0.2in}
\captionnamefont{\footnotesize\sffamily\bfseries}
\captiontitlefont{\footnotesize\sffamily}

%define column options
\setlength{\columnseprule}{0pt}
\def\columnseprulecolor{\color{columnlinecol}}

%define section title features
\setsubsubsecheadstyle{\small\color{sectitletextcol}\textbf}% Set \section style
\setsecnumformat{}
\def\sectionmark#1{\markboth{#1}{#1}}

%%%%%%%%%%%% TCOLORBOXES TO THE RESCUE %%%%%%%%%%%%%%%%%%%%
%Title Box
\newtcolorbox{topbox}{
enhanced,
colback=titleboxbgcol,
colframe=titleboxbordercol,
halign=center,
boxrule=0.25cm,
sharp corners=all,
 overlay={
    \node[anchor=south west]
      at ([xshift=1cm,yshift=1cm]frame.south west)
       {\includegraphics[width=3cm]{Figures/posterdownlogo}};
    \node[anchor=south east]
      at ([xshift=-1cm,yshift=1cm]frame.south east)
       {\includegraphics[width=3cm]{Figures/oulogo}};}

}
%Body Section Title Box
\newtcolorbox{myboxstuff}[1][]{
code={\parindent=0em},
colframe=sectitlebordercol,
nobeforeafter,
left skip=0pt,
valign=center,
halign=center,
fontupper=\Large\bfseries,
colupper=sectitletextcol,
boxrule=1mm,
colback=sectitlebgcol,
sharp corners=uphill, #1}
\newcommand{\mybox}[1]{%
\begin{myboxstuff}
\strut #1
\end{myboxstuff}%
}
\makeheadstyles{MyBox}{
    \setsecheadstyle{\mybox}
}
\headstyles{MyBox}\makepagestyle{MyBox}
%-----------------------------------------------------
%Make sure that the page is empty of any preset items from memoir
\thispagestyle{empty}

%biblatex options
\RequirePackage[sorting=none,backend=biber]{biblatex}
\renewcommand*{\bibfont}{\small} %% SC
\bibliography{MyLibrary}
\defbibheading{bibliography}[\bibname]{%
\setlength\bibitemsep{0.8\itemsep} %% SC
\section*{#1}%
\markboth{#1}{#1}}
\AtBeginDocument{%
  \renewcommand{\bibname}{References}
}

%Remove section numbering & set 2nd level header as first level
%to avoid the automatic new page generated from memoir chapter
%formatting
\counterwithout{section}{chapter}
\makechapterstyle{mydefault}{
\addtocounter{secnumdepth}{2}
\setsecheadstyle{\mybox}
\setsubsecheadstyle{\itshape}
\setsubsubsecheadstyle{\itshape}
}

%set the chapterstyle
\chapterstyle{mydefault}

%define column spacing
\setlength\columnsep{0.1cm}

%spacing params
\setlength\parindent{0em}
\setlength\parskip{0em}
\setlength\hangparas{0}

%spacing after section head title
\setaftersecskip{0em}
\setbeforesecskip{1.5em}
\setlength\textfloatsep{0in}
\setlength\floatsep{0in}
\setlength\intextsep{0in}

\setstocksize{420mm}{297mm}
\settrimmedsize{\stockheight}{\stockwidth}{*}
\settypeblocksize{420mm}{297mm}{*}
\setlrmargins{*}{*}{1}
\setulmarginsandblock{2.5cm}{*}{*}
\setmarginnotes{0em}{0cm}{0cm}
\setlength{\footskip}{0cm}
\setlength{\footnotesep}{0cm}
\setlength{\headheight}{0pt}
\setlength{\headsep}{0pt}
\setlength{\trimtop}{0pt}
\setlength{\trimedge}{0pt}
\setlength{\uppermargin}{0pt}
\checkandfixthelayout

%Footnote to white
\RequirePackage{footmisc}
\def\footnotelayout{\centering\color{footnotetextcol}}

% see https://stackoverflow.com/a/47122900

% choose font family
\RequirePackage{palatino}

% define the BODYBGCOL
\newpagecolor{bodybgcol}

%sets footnote to be white hopefully
\renewcommand\footnoterule{}
\renewcommand{\thempfootnote}{\footnotesize\color{footnotetextcol}{\arabic{mpfootnote}}}

%-------------- Begin Document -------------------%
\begin{document}

%-------------- Title Box Start ------------------%
%tcolorbox allows for pictures hopefully
\begin{topbox}
  \color{titletextcol}
  \vspace{0.5in}
  \Huge{\fontfamily{phv}\selectfont Otago University Default Landscape}  \\[0.3in]  %% SC
  \color{authortextcol} \Large{Matt Bixley\textsuperscript{1} Author Two\textsuperscript{2}} \\[0.2in] %% SC
  \color{affiliationtextcol} \large{\textsuperscript{1}Department of Biochemistry, University of Otago;
\textsuperscript{2}Deparment of Another Institution, Institution
University} %% SC
  \vspace{1cm}
\end{topbox}
%--------------- Title Box End -------------------%
%----------------- Body Start --------------------%
% Begin body of poster
\begin{adjmulticols*}{3}{0.1cm}{0.1cm}
\normalsize{  %% SC
\color{bodytextcol}
\section{Introduction}\label{introduction}

Welcome to \texttt{posterdown} ! This is my attempt to provide a
semi-smooth workflow for those who wish to take their \texttt{RMarkdown}
skills to the conference world. Many creature comforts from
\texttt{RMarkdown} are available in this package such as
\texttt{Markdown} section notation, figure captioning, and even
citations like this one \autocite{holden_identifying_2012} The rest of
this example poster will show how you can insert typical conference
poster features into your own document.

\section{Study Site}\label{study-site}

Here is a map made to show the study site using \texttt{ggplot2},
\texttt{ggspatial}, and \texttt{sf} and you can even reference this with
a hyperlink, this will take you to \textbf{Figure \ref{mymapfig}}. Lorem
ipsum dolor sit amet, \autocite{middleton_geological_nodate} consectetur
adipiscing elit, sed do eiusmod tempor incididunt ut labore et dolore
magna aliqua. \vspace{1cm}

\begin{figure}

{\centering \includegraphics[width=0.5\linewidth]{OtagoPosterPortrait_files/figure-latex/unnamed-chunk-2-1} 

}

\caption{This is a map of Canada, the ggspatial package is great for GIS folks in R! \label{mymapfig}}\label{fig:unnamed-chunk-2}
\end{figure}

\lipsum[4]

\section{Objectives}\label{objectives}

\begin{enumerate}
\def\labelenumi{\arabic{enumi}.}
\tightlist
\item
  Easy to use reproducible poster design.
\item
  Integration with \texttt{RMarkdown}.
\item
  Easy transition from \texttt{posterdown} to \texttt{thesisdown} or
  \texttt{rticles}
\end{enumerate}

\section{Methods}\label{methods}

This package uses the same workflow approach as the \texttt{RMarkdown}
you know and love. Basically it goes from RMarkdown \textgreater{} Knitr
\textgreater{} Markdown \textgreater{} Pandoc \textgreater{} Latex
\textgreater{} PDF.

\lipsum[2]

\section{Results}\label{results}

Usually you want to have a nice table displaying some important results
that you have calcualated. In posterdown this is as easy as using the
\texttt{kable} table formatting you are probably use to as per typical
\texttt{RMarkdown} formatting. I suggesting checking out the
\texttt{kableExtra} package and its in depth documentation on
customizing these tables found
\href{https://haozhu233.github.io/kableExtra/awesome_table_in_pdf.pdf}{here}.

\vspace{0.2cm}

\begin{table}[H]

\caption{\label{tab:unnamed-chunk-3}Tables are a breeze with Kable and Kable extra package!}
\centering
\fontsize{10}{12}\selectfont
\begin{tabu} to \linewidth {>{\centering}X>{\centering}X>{\centering}X>{\centering}X>{\centering}X}
\toprule
Sepal.Length & Sepal.Width & Petal.Length & Petal.Width & Species\\
\midrule
\rowcolor{gray!6}  5.1 & 3.5 & 1.4 & 0.2 & setosa\\
4.9 & 3.0 & 1.4 & 0.2 & setosa\\
\rowcolor{gray!6}  4.7 & 3.2 & 1.3 & 0.2 & setosa\\
4.6 & 3.1 & 1.5 & 0.2 & setosa\\
\bottomrule
\end{tabu}
\end{table}

\begin{figure}

{\centering \includegraphics[width=0.8\linewidth]{OtagoPosterPortrait_files/figure-latex/unnamed-chunk-4-1} 

}

\caption{Using ggplot and patchwork to generate a layout of multiple plots in one figure. The iris dataset was used to generate (a) a line graph, (b) a scatterplot, and (c) a boxplot all together!}\label{fig:unnamed-chunk-4}
\end{figure}

\lipsum[1]

\section{Next Steps}\label{next-steps}

There is still \textbf{A LOT} of work to do on this package which
include (but are note limited to):

\begin{itemize}
\tightlist
\item
  Better softcoding for front end user options in YAML
\item
  Images in the title section for logo placement which is a common
  attribut to posters as far as I have come to know.
\item
  Figure out compatiability with \texttt{natbib} which wasn't working
  during the initial set up.
\item
  MUCH BETTER PACKAGE DOCUMENTATION. For example, there is nothing in
  the README\ldots{}
\item
  Include References section only if initiated by the user like in
  RMarkdown.
\end{itemize}

\small\printbibliography
}
\end{adjmulticols*}
%------------------ Body End ---------------------%
%end the poster
\end{document}

